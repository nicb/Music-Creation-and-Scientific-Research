%
% $Id: slides.tex 7 2014-02-04 18:16:40Z nicb $
%
% Copyright (C) 2006 Nicola Bernardini nicb@sme-ccppd.org
% 
% This work is licensed under a Creative Commons License, and specifically the
% 
%   Creative Commons Attribution-ShareAlike 2.5 License
%   http://creativecommons.org/licenses/by-sa/2.5/legalcode
% 
% Check http://www.creativecommons.org/ for more information on
% Creative Commons Licenses and the Creative Commons Project.
%
% Set the macros below to whatever is appropriate in a given context
%

\newcommand{\topdir}{../../..}
\newcommand{\imagedir}{\topdir/images}
\newcommand{\exampledir}{\topdir/examples}
\newcommand{\templatedir}{\topdir/../latex-templates/beamer/sme-ccppd}
\documentclass[compress]{beamer}

\usepackage{beamerthemeSME-CCPPD}
\usepackage{beamercolorthemeSME-CCPPD}
\usepackage{beamerinnerthemeSME-CCPPD}

\usepackage{colortbl}
\usepackage[italian]{babel}
\usepackage{pgf}
\usepackage{xspace}

\usepackage{multimedia}
\usepackage{xmpmulti}
\usepackage{hyperref}
\usepackage[nofancy]{svninfo}
\newcommand{\rcstag}{v.\svnInfoRevision\ \ \svnInfoDate\xspace}
\usepackage{gensymb}

\newcommand{\hhref}[1]{\href{#1}{#1}}

\pgfdeclareimage[width=16pt]{speaker}{\imagedir/speaker}
\newcommand{\esdplay}[2]{\href{run: esdplay \exampledir/#1}{#2 \pgfuseimage{speaker}}}
\newcommand{\oggplay}[2]{\href{run: ogg123 -d esd \exampledir/#1}{#2 \pgfuseimage{speaker}}}

\newcommand{\cpyear}{2008}
\newcommand{\cpholder}{Nicola Bernardini}
\newcommand{\cpholderemail}{nicb@sme-ccppd.org}

% Use some nice templates

%\beamertemplateshadingbackground{red!10}{structure!10}
\beamertemplatetransparentcovereddynamic
\beamertemplateballitem
\beamertemplatenumberedballsectiontoc

% My colors
\definecolor{notdone}{gray}{0.35}

%\usecolortheme[named=MyColor]{structure}
%\usecolortheme[named=MyColor]{structure}
\beamertemplateshadingbackground{white!10}{white!10}

\title[Creativit\`a Musicale e Ricerca Scientifica]
{%
    Creativit\`a Musicale e Ricerca Scientifica\\
	{\tiny (\rcstag)}
}

\author{%
	Nicola Bernardini\\
    \href{mailto:\cpholderemail}{\cpholderemail}
}
\institute[SME-CCPPD]%
{%
	\href{http://www.conservatoriopollini.it}
		 {Conservatorio di Musica ``C.Pollini'' -- Padova}
}
\date[Padova, 13/11/2008]{Padova, Universit\`a di Padova -- DEI, 13/11/2008}

\begin{document}
\svnInfo $Id: slides.tex 7 2014-02-04 18:16:40Z nicb $
\newcounter{ms}
  
\begin{frame}
	\titlepage
\end{frame}

\begin{frame}
    
    \begin{columns}[T]
        \begin{column}{0.7\textwidth}
			    \begin{quote}
			        Per quanto la crescita dell'orchestra sia legata al caso delle
			        circostanze, delle condizioni, o dello sviluppo degli strumenti,
			        la vera spinta dietro alla sua evoluzione \`e stata dopotutto l'arte
			        musicale stessa e la sua domanda, sempre crescente e sempre pi\`u insistente,
			        di mezzi di espressione adeguati.%
			    \end{quote}
			    \begin{flushright}
			        Adam Carse,\\\emph{The History of Orchestration},\\p.7 (1925)
			    \end{flushright}
        \end{column}
        \begin{column}{0.3\textwidth}
            \begin{center}
                \pgfimage[width=\columnwidth]{\imagedir/carse-cover}
            \end{center}
        \end{column}
    \end{columns}

\end{frame}
  
\section{Introduzione}

\begin{frame}
    \frametitle{Il rapporto tra Musica e Innovazione Tecnologico/Scientifica}

    \begin{itemize}[<+- | alert@+->]
        \item Luogo comune: \`e l'innovazione tecnologica a spostare le
            frontiere dell'espressione musicale
        \item Questo \`e vero solo in parte
        \item In realt\`a, i processi sono due:
            \begin{enumerate}[<+- | alert@+->]
                \item I musicisti \emph{utilizzano} innovazioni tecnologiche
                    quando queste sono ben consolidate e, tutto sommato, in
                    modo poco innovativo
                \item I musicisti formulano in continuazione richieste alla
                    tecnologia (e -- in definitiva -- alla scienza) per
                    riuscire a raggiungere i propri obbiettivi espressivi ed
                    estetici
            \end{enumerate}
    \end{itemize}

\end{frame}

\section[Innovazione $\Rightarrow$ Musica]{L'Innovazione produce Musica}

\subsection{Berio -- {\it THEMA}}

\begin{frame}
    \frametitle{Luciano Berio -- \emph{THEMA -- Omaggio a Joyce}} 

    \begin{columns}[T]
        \begin{column}{0.7\textwidth}
			    \begin{itemize}
			        \item \emph{Blblblblooming}:
			
			            \begin{itemize}
			                \item Un anello veloce (nota pi\`u acuta)
			                \item Un anello veloce (nota pi\`u grave)
			                \item Un anello normale (ad altezza nominale)
			            \end{itemize}
			
			        \item \esdplay{Blblblbooming.wav}{Risultato:}
			
			    \end{itemize}
        \end{column}
        \begin{column}{0.3\textwidth}
            \pgfimage[width=0.9\columnwidth]{\imagedir/berio}
        \end{column}
    \end{columns}
        
\end{frame}

\subsection{Nono -- {\it Prae--Ludium}}

\setcounter{ms}{0}
\refstepcounter{ms}
\begin{frame}
    \frametitle{Luigi Nono -- \emph{Post--Prae Ludium per Donau} (\arabic{ms})} 

    \begin{center}
       \pgfimage[width=0.15\textwidth]{\imagedir/luigi_nono}\\[0.5\baselineskip]
       \pgfimage[width=\textwidth]{\imagedir/Nono-PPPD} 
    \end{center}

\end{frame}

\refstepcounter{ms}
\begin{frame}
    \frametitle{Luigi Nono -- \emph{Post--Prae Ludium per Donau} (\arabic{ms})} 

    \begin{columns}
        \begin{column}{0.9\textwidth}
            \vspace{-0.6cm}
            \begin{figure}
                \pgfimage[height=0.7\textheight]{\imagedir/Nono-PPPD-schema} 
                \caption{\emph{Post--Prae Ludium per Donau}, schema esecutivo di Alvise Vidolin}
            \end{figure}
        \end{column}
        \begin{column}{0.1\textwidth}
            \esdplay{Delay4_A_misto_rotto_Rev.wav}{}
        \end{column}
    \end{columns}

\end{frame}

\section[Musica $\Rightarrow$ Innovazione]{La Musica produce Innovazione}

\subsection[Classici]{Esempi Classici}

\begin{frame}
    \frametitle{Un esempio classico di stimolo dell'innovazione}

    \begin{columns}[T]
        \begin{column}{0.2\textwidth}
            \pgfimage[width=\columnwidth]{\imagedir/wagner-01} 
        \end{column}
        \begin{column}{0.5\textwidth}
            Richard Wagner, tornato in Germania dopo l'esilio nel 1861, si
            stabilisce a Biebrich, dove risiede anche Wilhelm Heckel,
            costruttore di fagotti. Dallo stretto rapporto che si stringe tra i due
            nascer\`a il \emph{Heckelphon} (utilizzato poi da Strauss nella
            \emph{Salom\'e}) e si perfezioner\`a il \emph{controfagotto} (utilizzato nel
            \emph{Parsifal}).
        \end{column}
        \begin{column}{0.05\textwidth}
            \pgfimage[height=0.8\textheight]{\imagedir/heckelphon_big}
        \end{column}
        \begin{column}{0.1\textwidth}
            \pgfimage[height=0.8\textheight]{\imagedir/kontrafagott_big}
        \end{column}
    \end{columns}
\end{frame}

\subsection[4X]{IRCAM 4X}

\begin{frame}
    \frametitle{Berio, Di Giugno e la 4X} 

    \begin{columns}[T]
        \begin{column}{0.6\textwidth}
            Berio invita il fisico Giuseppe Di Giugno all'IRCAM chiedendogli,
            nel 1976, un ``sintetizzatore con 1000 oscillatori''. Il
            sintetizzatore (denominato \emph{4X}) \`e pronto nel 1981.
            Nel frattempo, Berio \`e partito per altri lidi. 
            La \emph{4X} verr\`a poi industrializzata ed impiegata dall'industria aeronautica 
            per le simulazioni degli aerei Airbus.
        \end{column}
        \begin{column}{0.4\textwidth}
            \pgfimage[width=0.9\columnwidth]{\imagedir/boulez+di_giugno-001}\\
            \pgfimage[width=0.9\columnwidth]{\imagedir/4X-2}
        \end{column}
    \end{columns}
\end{frame}

\subsection[DX7]{Chowning e il DX7}

\begin{frame}
    \frametitle{Chowning e lo Yamaha DX7}
    
    \begin{columns}[T]
        \begin{column}{0.6\textwidth}
            Il compositore John Chowning elabora la sintesi dei suoni per
            modulazione di frequenza nel '67-'68 per comporre i suoi lavori.
            La tecnica verr\`a poi utilizzata nei sintetizzatori \emph{Yamaha
            DX7}, che rivoluzioneranno il mercato degli strumenti musicali
            negli anni '80.
        \end{column}
        \begin{column}{0.4\textwidth}
            \begin{center}
                \pgfimage[width=0.6\columnwidth]{\imagedir/chowning}
            \end{center}
            \hspace{-4cm}\pgfimage[width=1.9\columnwidth]{\imagedir/dx7}
        \end{column}
    \end{columns}

\end{frame}

\subsection[Morphing]{Berio e il {\it morphing}}

\begin{frame}
    \frametitle{Berio e il \emph{Morphing}}

    \begin{columns}[T]
        \begin{column}{0.7\textwidth}
            \begin{itemize}
                \item Berio era affascinato dall'idea del \emph{morphing}
                    timbrico tra uno strumento e l'altro
                \item Esperimento fallito: \emph{Chemins V} (poi ritirato) nel 1980
                \item Esperimento pi\`u riuscito: l'introduzione di \emph{Ofanim}
                \item La ricerca continua\dots
            \end{itemize}
        \end{column}
        \begin{column}{0.3\textwidth}
            \pgfimage[width=0.9\columnwidth]{\imagedir/berio-02}
        \end{column}
    \end{columns}

\end{frame}

\subsection[Spazializzazione]{La Spazializzazione Espressiva}

\setcounter{ms}{0}

\refstepcounter{ms}
\begin{frame}
    \frametitle{La Spazializzazione Espressiva (\arabic{ms})}

    \begin{columns}[T]
        \begin{column}{0.7\textwidth}
            Nel 2002, il compositore italiano Adriano Guarnieri chiese di
            poter rendere in maniera pi\`u espressiva la spazializzazione del
            quartetto di tromboni nell'esecuzione di \emph{Medea}.
            In seno al Centro di Sonologia Computazionale di Padova venne
            realizzato un sistema di ``spazializzazione espressiva'' unico nel
            suo genere.
        \end{column}
        \begin{column}{0.3\textwidth}
            \pgfimage[width=\columnwidth]{\imagedir/guarnieri-02} 
        \end{column}
    \end{columns}
    
\end{frame}

\refstepcounter{ms}
\begin{frame}
    \frametitle{La Spazializzazione Espressiva (\arabic{ms})}

    \begin{center}
        \begin{figure}
            \begin{center}
                \pgfimage[height=0.7\textheight]{\imagedir/eyesweb-processing}
            \end{center}
            \caption{Schema della ``spazializzazione espressiva'' per \emph{Medea} (Amalia de G\"otzen)}
            \label{fig:medea processing}
        \end{figure}
    \end{center}
    
\end{frame}

\subsection[Guarnieri]{Le Partiture di Adriano Guarnieri}

\setcounter{ms}{0}

\refstepcounter{ms}
\begin{frame}
    \frametitle{Le Partiture di Adriano Guarnieri (\arabic{ms})}

    \begin{columns}[T]
        \begin{column}{0.7\textwidth}
				    \begin{itemize}[<+- | alert@+->]
				       \item Molti sono i problemi aperti\dots
				       \item \dots ad esempio quelli delle partiture di Adriano
				                    Guarnieri
				    \end{itemize}
        \end{column}
        \begin{column}{0.3\textwidth}
            \pgfimage[width=\columnwidth]{\imagedir/guarnieri} 
        \end{column}
    \end{columns}
    
\end{frame}

\refstepcounter{ms}
\begin{frame}
    \frametitle{Le Partiture di Adriano Guarnieri (\arabic{ms})}

    \begin{columns}[T]
        \begin{column}{0.9\textwidth}
            \href{run: display \imagedir/Guarnieri-Passione-Lettera_M}
                {\pgfimage[width=\columnwidth]{\imagedir/Guarnieri-Passione-Lettera_M}}
        \end{column}
        \begin{column}{0.1\textwidth}
            \oggplay{Guarnieri-Passione_Lettera_M.ogg}{}
        \end{column}
    \end{columns}

\end{frame}

\begin{frame}
    \frametitle{(Le analisi di Heinrich Schenker)}

    \begin{center}
        \begin{figure}
            \pgfimage[height=0.7\textheight]{\imagedir/Schenker-Chopin_Etude_F_Major_op.10-n.8}
            \caption{Schenker -- Analisi dello \emph{Studio in Fa Maggiore} di Fryderyk Chopin op.10 n.8}
        \end{figure}
    \end{center}

\end{frame}

\subsection[Scelsi]{I nastri di Giacinto Scelsi}

\setcounter{ms}{0}
\refstepcounter{ms}
\begin{frame}
    \frametitle{I Nastri di Giacinto Scelsi (\arabic{ms})}
    
    \begin{columns}[T]
        \begin{column}{0.3\textwidth}
            \pgfimage[width=\columnwidth]{\imagedir/Scelsi}
        \end{column}
        \begin{column}{0.7\textwidth}
            \begin{itemize}[<+- | alert@+->]
                \item Scelsi, compositore italiano vissuto tra il 1905 ed il
                    1988, usava il nastro magnetico come
                    ``quaderno di schizzi'' per le sue composizioni\dots
                \item Capire la traiettoria creativa di queste ultime
                    costituirebbe un successo nell'ambito del \emph{data
                    mining}
            \end{itemize}
        \end{column}
    \end{columns}

\end{frame}

\refstepcounter{ms}
\begin{frame}
    \frametitle{I Nastri di Giacinto Scelsi (\arabic{ms})}
    
    \begin{center}
        \pgfimage[height=0.8\textheight]{\imagedir/CONTAKTE-concept} 
    \end{center}

\end{frame}

\section[Come]{Come si realizza questo sodalizio?}

\subsection{Problemi}

\begin{frame}
    \frametitle{Creazione Musicale e Ricerca Scientifica -- i Problemi}

    \begin{itemize}[<+- | alert@+->]
        \item La musica risponde ad imperativi estetici, mentre la scienza
            risponde a categorie razionali
        \item La musica ha (in genere) scadenze improrogabili e molto
            pressanti, mentre la ricerca non pu\`o garantire i propri
            risultati
        \item Il compositore \`e (in genere) un autarchico, mentre la ricerca
            deve basarsi su collaborazioni e condivisioni dei propri risultati
        \item Ciascun idioma musicale contemporaneo \`e fortemente
            caratterizzato e specifico, mentre la ricerca deve adottare
            terminologie e notazioni universalmente riconosciute
        \item \ldots
    \end{itemize}

\end{frame}

\subsection{Soluzioni}

\begin{frame}
    \frametitle{Binari Paralleli, Ambienti e Mediatori}

    \begin{itemize}[<+- | alert@+->]
        \item Composizione Musicale e Ricerca Scientifica non devono dipendere
            strettamente l'una dall'altra: esse devono viaggiare su due binari
            paralleli
        \item Gli ambienti ideali per un incontro non possono essere i laboratori
            convenzionali: sono un po' laboratori, un po' studi, un po'
            salotti
        \item Sono ambienti in cui si fa \emph{cultura}, non si fa \emph{intrattenimento}
        \item Spesso, l'incontro tra Compositori e Scienziati avviene con
            l'ausilio di figure di mediazione -- i ``registi'' dell'opera in
            questione
    \end{itemize}
    
\end{frame}

\begin{frame}
    \frametitle{Come si arriva all'Innovazione Tecnologica e Scientifica?}

    \begin{itemize}[<+- | alert@+->]
        \item Il processo che porta dalla Creazione Musicale all'Innovazione
            Tecnologica \`e ancora in larga parte oscuro (informazione
            \emph{implicita})
        \item L'industria e l'economia sono ora interessate a capire quali
            sono le modalit\`a ed i processi che possano favorire un salto
            qualitativo dell'innovazione
        \item Studio etnografico di queste comunit\`a ``miste''
    \end{itemize}
    
\end{frame}

\begin{frame}
    \begin{center}
        {\Huge FINE}\\[2\baselineskip]
        Grazie per l'ascolto.
    \end{center}
\end{frame}

\end{document}
