%
% $Id: slides.tex 152 2015-06-24 04:46:28Z nicb $
%
% Copyright (C) 2006 Nicola Bernardini nicb@sme-ccppd.org
% 
% This work is licensed under a Creative Commons License, and specifically the
% 
%   Creative Commons Attribution-ShareAlike 2.5 License
%   http://creativecommons.org/licenses/by-sa/2.5/legalcode
% 
% Check http://www.creativecommons.org/ for more information on
% Creative Commons Licenses and the Creative Commons Project.
%
% Set the macros below to whatever is appropriate in a given context
%

\newcommand{\topdir}{../../..}
\newcommand{\imagedir}{\topdir/images}
\newcommand{\exampledir}{\topdir/examples}
\newcommand{\templatedir}{\topdir/../latex-templates/beamer/smerm}
\newcommand{\commondir}{../common}
\documentclass[compress]{beamer}

\usepackage{beamerthemeSMERM}
\usepackage{beamercolorthemeSMERM}
\usepackage{beamerinnerthemeSMERM}

\usepackage{colortbl}
\usepackage[italian]{babel}
\usepackage{pgf}
\usepackage{xspace}

\usepackage{multimedia}
\usepackage{xmpmulti}
\usepackage{hyperref}
\usepackage[nofancy]{svninfo}
\usepackage{gensymb}
\usepackage{gitinfo2}

\newcommand{\cpyear}{2015}
\newcommand{\cpholder}{Nicola Bernardini}
\newcommand{\cpholderemail}{n.bernardini@conservatoriosantacecilia.it}

% Use some nice templates

%\beamertemplateshadingbackground{red!10}{structure!10}
\beamertemplatetransparentcovereddynamic
\beamertemplateballitem
\beamertemplatenumberedballsectiontoc

% My colors
\definecolor{notdone}{gray}{0.35}

%\usecolortheme[named=MyColor]{structure}
%\usecolortheme[named=MyColor]{structure}
\beamertemplateshadingbackground{white!10}{white!10}

\input{\templatedir/macros}

\title[Creativit\`a Musicale e Ricerca Scientifica]
{%
    Creativit\`a Musicale e Ricerca Scientifica\\
	{\tiny (\rcstag)}
}

\author{%
	Nicola Bernardini\\
    \href{mailto:\cpholderemail}{\cpholderemail}
}
\institute[SMERM]%
{%
	\href{http://www.conservatoriosantacecilia.it}
		 {Conservatorio di Musica ``S.Cecilia'' -- Roma}
}
\date[Roma, 24/06/2015]{La Sapienza Universit\`a di Roma\\XII Workshop Tecnologie per la Musica}

\begin{document}
\svnInfo $Id: slides.tex 152 2015-06-24 04:46:28Z nicb $
\newcounter{ms}
  
\begin{frame}
	\titlepage
\end{frame}

\begin{frame}
    
    \begin{columns}[T]
        \begin{column}{0.7\textwidth}
			    \begin{quote}
						However much the growth of the orchestra has been
						the toy of circumstances, conditions, or the mechanical or
						technical development of instruments, the real driving force
						behind such evolution is after all the insistently growing demand
						of musical art for fit means of expression.
			    \end{quote}
			    \begin{flushright}
			        Adam Carse,\\\emph{The History of Orchestration},\\p.7 (1925)
			    \end{flushright}
        \end{column}
        \begin{column}{0.3\textwidth}
            \begin{center}
                \pgfimage[width=\columnwidth]{\imagedir/carse-cover}
            \end{center}
        \end{column}
    \end{columns}

\end{frame}
  


\section{Introduzione}

\begin{frame}
    \frametitle<+- | alert@+->{The relationship between Music and Scientific/Technological Innovation}

    \begin{itemize}[<+- | alert@+->]
        \item Commonplace: it is technology innovation that
								pushes the boundaries of musical expression
        \item This is only partly true
        \item In fact, two processes take place:
            \begin{enumerate}[<+- | alert@+->]
								\item Musicians use technological innovation once
												it is a robust and solid reality and, all in all,
												not in a particularly innovative way
								\item Musicians continuously express requests to technology
												(and -- ultimately -- to science) to be able
												to reach their own expressive and aesthetic objectives
            \end{enumerate}
    \end{itemize}

\end{frame}



\section[Innovazione $\Rightarrow$ Musica]{L'Innovazione produce Musica}

\subsection{Berio -- {\it THEMA}}

\begin{frame}
    \frametitle{\normalsize Luciano Berio -- \emph{THEMA -- Omaggio a Joyce} (1958)} 

    \begin{columns}[T]
        \begin{column}{0.7\textwidth}
          \begin{itemize}
              \item \emph{Blblblblooming}:
      
                  \begin{itemize}
                      \item A faster loop (higher note)
                      \item A faster loop (lower note)
                      \item A normal loop (nominal pitch)
                  \end{itemize}
      
              \item Risultato: \listento{\exampledir/Blblblbooming.wav}{}
      
          \end{itemize}
        \end{column}
        \begin{column}{0.3\textwidth}
            \pgfimage[width=0.9\columnwidth]{\imagedir/berio}
        \end{column}
    \end{columns}
        
\end{frame}


\subsection{Nono -- {\it Donau}}

\setcounter{ms}{0}
\refstepcounter{ms}
\begin{frame}
    \frametitle{\normalsize Luigi Nono -- \emph{Post--Prae Ludium per Donau} (1987) (\arabic{ms})} 

    \begin{center}
       \pgfimage[width=0.15\textwidth]{\imagedir/luigi_nono}\\[0.5\baselineskip]
       \pgfimage[width=0.9\textwidth]{\imagedir/Nono-PPPD} 
    \end{center}

\end{frame}

\refstepcounter{ms}
\begin{frame}
    \frametitle{\normalsize Luigi Nono -- \emph{Post--Prae Ludium per Donau} (1987) (\arabic{ms})} 

     \vspace{-0.6cm}
     \begin{figure}
        \pgfimage[height=0.6\textheight]{\imagedir/Nono-PPPD-schema} 
				\caption{\listento{\exampledir/Delay4_A_misto_rotto_Rev.wav}{\alert{\emph{Post--Prae Ludium per Donau}, performance schema by Alvise Vidolin}}}
     \end{figure}

\end{frame}



\section[Musica $\Rightarrow$ Innovazione]{La Musica produce Innovazione}

\subsection[Classici]{Esempi Classici}

\begin{frame}
		\frametitle{A classical example of innovation stimulus}

    \begin{columns}[T]
        \begin{column}{0.2\textwidth}
            \pgfimage[width=\columnwidth]{\imagedir/wagner-01} 
        \end{column}
        \begin{column}{0.5\textwidth}
					\begin{small}
						\alert{Richard Wagner, back to Germany after his exile in 1861,
						settles in Biebrich where Wilhelm Heckel, the bassoon builder, lives.
						The close relationship between composer and instrument maker
						will generate the \emph{Heckelphon} (which will be used later on
						by Strauss in his \emph{Salom\'e}) and a considerable improvement
						of the \emph{contrabassoon} (which will be used in
						\emph{Parsifal}).}
					\end{small}
        \end{column}
        \begin{column}{0.05\textwidth}
            \pgfimage[height=0.8\textheight]{\imagedir/heckelphon_big}
        \end{column}
        \begin{column}{0.1\textwidth}
            \pgfimage[height=0.8\textheight]{\imagedir/kontrafagott_big}
        \end{column}
    \end{columns}
\end{frame}



\subsection[4X]{IRCAM 4X}

\begin{frame}
    \frametitle{Berio, Di Giugno and the 4X} 

    \begin{columns}[T]
        \begin{column}{0.6\textwidth}
						Berio invites at IRCAM the physicist Giuseppe Di Giugno
						asking him, in 1976, a ``synthesizer with 1000 oscillators''.
						Such synthesizer (named \emph{4X}) is finally ready in 1981.
						In the meantime, Berio is gone.
						However, the \emph{4X} will then be used in countless musical
						productions and will be finally industrialized by the aviation
						industry for the sound simulation of Airbus jets.
        \end{column}
        \begin{column}{0.4\textwidth}
            \pgfimage[width=0.9\columnwidth]{\imagedir/boulez+di_giugno-001}\\
            \pgfimage[width=0.9\columnwidth]{\imagedir/4X-2}
        \end{column}
    \end{columns}
\end{frame}


\subsection[DX7]{Chowning e il DX7}

\begin{frame}
    \frametitle{Chowning e lo Yamaha DX7}
    
    \begin{columns}[T]
        \begin{column}{0.6\textwidth}
            Il compositore John Chowning elabora la sintesi dei suoni per
            modulazione di frequenza nel '67-'68 per comporre i suoi lavori.
            La tecnica verr\`a poi utilizzata nei sintetizzatori \emph{Yamaha
            DX7}, che rivoluzioneranno il mercato degli strumenti musicali
            negli anni '80.
        \end{column}
        \begin{column}{0.4\textwidth}
            \begin{center}
                \pgfimage[width=0.6\columnwidth]{\imagedir/chowning}
            \end{center}
            \hspace{-4cm}\pgfimage[width=1.8\columnwidth]{\imagedir/dx7}
        \end{column}
    \end{columns}

\end{frame}



\subsection[Morphing]{Berio e il {\it morphing}}

\begin{frame}
    \frametitle<+- | alert@+->{Berio and \emph{Morphing}}

    \begin{columns}[T]
        \begin{column}{0.7\textwidth}
            \begin{itemize}[<+- | alert@+->]
								\item Berio was fascinated by the idea of timbral \emph{morphing}
												between one instrument and another
								\item Failed experiment: \emph{Chemins V} in 1980 (withdrawn afterwards)
								\item Better success: the introduction of \emph{Ofanim} in 1989
								\item Research continues on this topic\ldots
            \end{itemize}
        \end{column}
        \begin{column}{0.3\textwidth}
            \pgfimage[width=0.9\columnwidth]{\imagedir/berio-02}
        \end{column}
    \end{columns}

\end{frame}



\subsection[Spazializzazione]{La Spazializzazione Espressiva}

\setcounter{ms}{0}
\refstepcounter{ms}
\begin{frame}
    \frametitle<+- | alert@+->{Expressive spatializazion (\arabic{ms})}

    \begin{columns}[T]
        \begin{column}{0.7\textwidth}
						In 2002, italian composer Adriano Guarnieri asked a more
						expressive spatial rendering of a trombone quartet within the
						performance of his opera \emph{Medea}.
					  The problem was studied within the Centro di Sonologia
						Computazionale of the University of Padova and an ``expressive
						spatialization system'', unique in its genre, was created.
        \end{column}
        \begin{column}{0.3\textwidth}
            \pgfimage[width=\columnwidth]{\imagedir/guarnieri-02} 
        \end{column}
    \end{columns}
    
\end{frame}

\refstepcounter{ms}
\begin{frame}
    \frametitle{Expressive spatialization (\arabic{ms})}

    \begin{center}
        \begin{figure}
            \begin{center}
                \pgfimage[height=0.6\textheight]{\imagedir/eyesweb-processing}
            \end{center}
            \caption{\emph{Medea}'s ``expressive spatialization'' system (Amalia de G\"otzen)}
            \label{fig:medea processing}
        \end{figure}
    \end{center}
    
\end{frame}


\subsection[Guarnieri]{Le Partiture di Adriano Guarnieri}

\setcounter{ms}{0}
\refstepcounter{ms}
\begin{frame}
    \frametitle<+- | alert@+->{Le Partiture di Adriano Guarnieri (\arabic{ms})}

    \begin{columns}[T]
        \begin{column}{0.7\textwidth}
				    \begin{itemize}[<+- | alert@+->]
				       \item Molti sono i problemi aperti\dots
				       \item \dots ad esempio quelli delle partiture di Adriano
				                    Guarnieri
				    \end{itemize}
        \end{column}
        \begin{column}{0.3\textwidth}
            \pgfimage[width=\columnwidth]{\imagedir/guarnieri} 
        \end{column}
    \end{columns}
    
\end{frame}

\refstepcounter{ms}
\begin{frame}
    \frametitle{Le Partiture di Adriano Guarnieri (\arabic{ms})}

		\vspace{-0.4cm}
		\begin{figure}[H!]
       \href{run:\imagedir/Guarnieri-Passione-Lettera_M.jpg}
             {\pgfimage[width=0.8\columnwidth]{\imagedir/Guarnieri-Passione-Lettera_M}}
			 \caption{\listento{\exampledir/Guarnieri-Passione_Lettera_M.ogg}{\emph{Passione secondo Matteo}, Adriano Guarnieri (letter M)}}
		\end{figure}

\end{frame}

\begin{frame}
    \frametitle{(Le analisi di Heinrich Schenker)}

    \begin{center}
        \begin{figure}
            \pgfimage[height=0.6\textheight]{\imagedir/Schenker-Chopin_Etude_F_Major_op.10-n.8}
            \caption{Schenker -- Analisi dello \emph{Studio in Fa Maggiore} di Fryderyk Chopin op.10 n.8}
        \end{figure}
    \end{center}

\end{frame}


\subsection[Scelsi]{I nastri di Giacinto Scelsi}

\setcounter{ms}{0}
\refstepcounter{ms}
\begin{frame}
    \frametitle<+- | alert@+->{Giacinto Scelsi's tapes (\arabic{ms})}
    
    \begin{columns}[T]
        \begin{column}{0.3\textwidth}
            \pgfimage[width=\columnwidth]{\imagedir/Scelsi}
        \end{column}
        \begin{column}{0.7\textwidth}
            \begin{itemize}[<+- | alert@+->]
								\item Italian composer Giacinto Scelsi (1905--1988)
												used magnetic recordings as ``sketchbooks''
												for his compositions\ldots
								\item To be able to understand his compositional trajectories 
												provides a formidable challenge in the \emph{data
												mining} field
            \end{itemize}
        \end{column}
    \end{columns}

\end{frame}

\refstepcounter{ms}
\begin{frame}
    \frametitle{Giacinto Scelsi's tapes (\arabic{ms})}
    
    \begin{center}
        \pgfimage[height=0.7\textheight]{\imagedir/CONTAKTE-concept} 
    \end{center}

\end{frame}



\section[Come]{Come si realizza questo sodalizio?}

\subsection{Problemi}

\begin{frame}
    \frametitle<+- | alert@+->{Creazione Musicale e Ricerca Scientifica -- i Problemi}

    \begin{itemize}[<+- | alert@+->]
        \item La musica risponde ad imperativi estetici, mentre la scienza
            risponde a categorie razionali
        \item La musica ha (in genere) scadenze improrogabili e molto
            pressanti, mentre la ricerca non pu\`o garantire i propri
            risultati
        \item Il compositore \`e (in genere) un autarchico, mentre la ricerca
            deve basarsi su collaborazioni e condivisioni dei propri risultati
        \item Ciascun idioma musicale contemporaneo \`e fortemente
            caratterizzato e specifico, mentre la ricerca deve adottare
            terminologie e notazioni universalmente riconosciute
        \item \ldots
    \end{itemize}

\end{frame}

\subsection{Soluzioni}

\begin{frame}
    \frametitle<+- | alert@+->{Binari Paralleli, Ambienti e Mediatori}

    \begin{itemize}[<+- | alert@+->]
        \item Composizione Musicale e Ricerca Scientifica non devono dipendere
            strettamente l'una dall'altra: esse devono viaggiare su due binari
            paralleli
        \item Gli ambienti ideali per un incontro non possono essere i laboratori
            convenzionali: sono un po' laboratori, un po' studi, un po'
            salotti
        \item Sono ambienti in cui si fa \emph{cultura}, non si fa \emph{intrattenimento}
        \item Spesso, l'incontro tra Compositori e Scienziati avviene con
            l'ausilio di figure di mediazione -- i ``registi'' dell'opera in
            questione
    \end{itemize}
    
\end{frame}

\begin{frame}
    \frametitle<+- | alert@+->{Come si arriva all'Innovazione Tecnologica e Scientifica?}

    \begin{itemize}[<+- | alert@+->]
        \item Il processo che porta dalla Creazione Musicale all'Innovazione
            Tecnologica \`e ancora in larga parte oscuro (informazione
            \emph{implicita})
        \item L'industria e l'economia dovrebbero essere interessate a capire quali
            sono le modalit\`a ed i processi che possano favorire un salto
            qualitativo dell'innovazione (non solo l'innovazione incrementale)
        \item Studio etnografico di queste comunit\`a ``miste''
    \end{itemize}
    
\end{frame}


\begin{frame}
    \begin{center}
        {\Huge FINE}\\[2\baselineskip]
        Grazie per l'ascolto.
    \end{center}
\end{frame}

\end{document}
