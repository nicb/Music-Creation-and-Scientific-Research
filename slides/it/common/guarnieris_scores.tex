\setcounter{ms}{0}
\refstepcounter{ms}
\begin{frame}
    \frametitle<+- | alert@+->{Le Partiture di Adriano Guarnieri (\arabic{ms})}

    \begin{columns}[T]
        \begin{column}{0.7\textwidth}
				    \begin{itemize}[<+- | alert@+->]
				       \item Molti sono i problemi aperti\dots
				       \item \dots ad esempio quelli delle partiture di Adriano
				                    Guarnieri
				    \end{itemize}
        \end{column}
        \begin{column}{0.3\textwidth}
            \pgfimage[width=\columnwidth]{\imagedir/guarnieri} 
        \end{column}
    \end{columns}
    
\end{frame}

\refstepcounter{ms}
\begin{frame}
    \frametitle{Le Partiture di Adriano Guarnieri (\arabic{ms})}

		\vspace{-0.4cm}
		\begin{figure}[H!]
       \href{run:\imagedir/Guarnieri-Passione-Lettera_M.jpg}
             {\pgfimage[width=0.8\columnwidth]{\imagedir/Guarnieri-Passione-Lettera_M}}
			 \caption{\listento{\exampledir/Guarnieri-Passione_Lettera_M.ogg}{\emph{Passione secondo Matteo}, Adriano Guarnieri (letter M)}}
		\end{figure}

\end{frame}

\begin{frame}
    \frametitle{(Le analisi di Heinrich Schenker)}

    \begin{center}
        \begin{figure}
            \pgfimage[height=0.6\textheight]{\imagedir/Schenker-Chopin_Etude_F_Major_op.10-n.8}
            \caption{Schenker -- Analisi dello \emph{Studio in Fa Maggiore} di Fryderyk Chopin op.10 n.8}
        \end{figure}
    \end{center}

\end{frame}
