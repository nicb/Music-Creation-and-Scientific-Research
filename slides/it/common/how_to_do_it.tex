\subsection{Problemi}

\begin{frame}
    \frametitle<+- | alert@+->{Creazione Musicale e Ricerca Scientifica -- i Problemi}

    \begin{itemize}[<+- | alert@+->]
        \item La musica risponde ad imperativi estetici, mentre la scienza
            risponde a categorie razionali
        \item La musica ha (in genere) scadenze improrogabili e molto
            pressanti, mentre la ricerca non pu\`o garantire i propri
            risultati
        \item Il compositore \`e (in genere) un autarchico, mentre la ricerca
            deve basarsi su collaborazioni e condivisioni dei propri risultati
        \item Ciascun idioma musicale contemporaneo \`e fortemente
            caratterizzato e specifico, mentre la ricerca deve adottare
            terminologie e notazioni universalmente riconosciute
        \item \ldots
    \end{itemize}

\end{frame}

\subsection{Soluzioni}

\begin{frame}
    \frametitle<+- | alert@+->{Binari Paralleli, Ambienti e Mediatori}

    \begin{itemize}[<+- | alert@+->]
        \item Composizione Musicale e Ricerca Scientifica non devono dipendere
            strettamente l'una dall'altra: esse devono viaggiare su due binari
            paralleli
        \item Gli ambienti ideali per un incontro non possono essere i laboratori
            convenzionali: sono un po' laboratori, un po' studi, un po'
            salotti
        \item Sono ambienti in cui si fa \emph{cultura}, non si fa \emph{intrattenimento}
        \item Spesso, l'incontro tra Compositori e Scienziati avviene con
            l'ausilio di figure di mediazione -- i ``registi'' dell'opera in
            questione
    \end{itemize}
    
\end{frame}

\begin{frame}
    \frametitle<+- | alert@+->{Come si arriva all'Innovazione Tecnologica e Scientifica?}

    \begin{itemize}[<+- | alert@+->]
        \item Il processo che porta dalla Creazione Musicale all'Innovazione
            Tecnologica \`e ancora in larga parte oscuro (informazione
            \emph{implicita})
        \item L'industria e l'economia dovrebbero essere interessate a capire quali
            sono le modalit\`a ed i processi che possano favorire un salto
            qualitativo dell'innovazione (non solo l'innovazione incrementale)
        \item Studio etnografico di queste comunit\`a ``miste''
    \end{itemize}
    
\end{frame}
