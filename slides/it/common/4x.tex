\begin{frame}
    \frametitle{Berio, Di Giugno e la 4X} 

    \begin{columns}[T]
        \begin{column}{0.6\textwidth}
            Berio invita il fisico Giuseppe Di Giugno all'IRCAM chiedendogli,
            nel 1976, un ``sintetizzatore con 1000 oscillatori''. Il
            sintetizzatore (denominato \emph{4X}) \`e pronto nel 1981.
            Nel frattempo, Berio \`e partito per altri lidi. 
            La \emph{4X} verr\`a poi industrializzata ed impiegata dall'industria aeronautica 
            per le simulazioni degli aerei Airbus.
        \end{column}
        \begin{column}{0.4\textwidth}
            \pgfimage[width=0.9\columnwidth]{\imagedir/boulez+di_giugno-001}\\
            \pgfimage[width=0.9\columnwidth]{\imagedir/4X-2}
        \end{column}
    \end{columns}
\end{frame}

