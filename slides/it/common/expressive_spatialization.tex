\setcounter{ms}{0}
\refstepcounter{ms}
\begin{frame}
    \frametitle<+- | alert@+->{La Spazializzazione Espressiva (\arabic{ms})}

    \begin{columns}[T]
        \begin{column}{0.7\textwidth}
            Nel 2002, il compositore italiano Adriano Guarnieri chiese di
            poter rendere in maniera pi\`u espressiva la spazializzazione del
            quartetto di tromboni nell'esecuzione di \emph{Medea}.
            In seno al Centro di Sonologia Computazionale di Padova venne
            realizzato un sistema di ``spazializzazione espressiva'' unico nel
            suo genere.
        \end{column}
        \begin{column}{0.3\textwidth}
            \pgfimage[width=\columnwidth]{\imagedir/guarnieri-02} 
        \end{column}
    \end{columns}
    
\end{frame}

\refstepcounter{ms}
\begin{frame}
    \frametitle{La Spazializzazione Espressiva (\arabic{ms})}

    \begin{center}
        \begin{figure}
            \begin{center}
                \pgfimage[height=0.6\textheight]{\imagedir/eyesweb-processing}
            \end{center}
            \caption{Schema della ``spazializzazione espressiva'' per \emph{Medea} (Amalia de G\"otzen)}
            \label{fig:medea processing}
        \end{figure}
    \end{center}
    
\end{frame}
