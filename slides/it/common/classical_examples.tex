\begin{frame}
    \frametitle{Un esempio classico di stimolo dell'innovazione}

    \begin{columns}[T]
        \begin{column}{0.2\textwidth}
            \pgfimage[width=\columnwidth]{\imagedir/wagner-01} 
        \end{column}
        \begin{column}{0.5\textwidth}
            Richard Wagner, tornato in Germania dopo l'esilio nel 1861, si
            stabilisce a Biebrich, dove risiede anche Wilhelm Heckel,
            costruttore di fagotti. Dallo stretto rapporto che si stringe tra i due
            nascer\`a il \emph{Heckelphon} (utilizzato poi da Strauss nella
            \emph{Salom\'e}) e si perfezioner\`a il \emph{controfagotto} (utilizzato nel
            \emph{Parsifal}).
        \end{column}
        \begin{column}{0.05\textwidth}
            \pgfimage[height=0.8\textheight]{\imagedir/heckelphon_big}
        \end{column}
        \begin{column}{0.1\textwidth}
            \pgfimage[height=0.8\textheight]{\imagedir/kontrafagott_big}
        \end{column}
    \end{columns}
\end{frame}

