\setcounter{ms}{0}
\refstepcounter{ms}
\begin{frame}
    \frametitle<+- | alert@+->{I Nastri di Giacinto Scelsi (\arabic{ms})}
    
    \begin{columns}[T]
        \begin{column}{0.3\textwidth}
            \pgfimage[width=\columnwidth]{\imagedir/Scelsi}
        \end{column}
        \begin{column}{0.7\textwidth}
            \begin{itemize}[<+- | alert@+->]
                \item Scelsi, compositore italiano vissuto tra il 1905 ed il
                    1988, usava il nastro magnetico come
                    ``quaderno di schizzi'' per le sue composizioni\dots
                \item Capire la traiettoria creativa di queste ultime
                    costituirebbe un successo nell'ambito del \emph{data
                    mining}
            \end{itemize}
        \end{column}
    \end{columns}

\end{frame}

\refstepcounter{ms}
\begin{frame}
    \frametitle{I Nastri di Giacinto Scelsi (\arabic{ms})}
    
    \begin{center}
        \pgfimage[height=0.7\textheight]{\imagedir/CONTAKTE-concept} 
    \end{center}

\end{frame}

