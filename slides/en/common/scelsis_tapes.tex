\setcounter{ms}{0}
\refstepcounter{ms}
\begin{frame}
    \frametitle<+- | alert@+->{The tapes by Giacinto Scelsi (\arabic{ms})}
    
    \begin{columns}[T]
        \begin{column}{0.3\textwidth}
            \pgfimage[width=\columnwidth]{\imagedir/Scelsi}
        \end{column}
        \begin{column}{0.7\textwidth}
            \begin{itemize}[<+- | alert@+->]
								\item Italian composer Giacinto Scelsi (1905--1988)
												used magnetic recordings as ``sketchbooks''
												for his compositions\ldots
								\item To be able to understand his compositional trajectories 
												provides a formidable challenge in the \emph{data
												mining} field
            \end{itemize}
        \end{column}
    \end{columns}

\end{frame}

\refstepcounter{ms}
\begin{frame}
    \frametitle{The tapes by Giacinto Scelsi (\arabic{ms})}
    
    \begin{center}
        \pgfimage[height=0.7\textheight]{\imagedir/CONTAKTE-concept} 
    \end{center}

\end{frame}

