\begin{frame}
		\frametitle{A classical example of innovation stimulus}

    \begin{columns}[T]
        \begin{column}{0.2\textwidth}
            \pgfimage[width=\columnwidth]{\imagedir/wagner-01} 
        \end{column}
        \begin{column}{0.5\textwidth}
						Richard Wagner, back to Germany after his exile in 1861,
						settles in Biebrich where Wilhelm Heckel, bassoon builder, lives.
						The close relationship between composer and instrument maker
						will generate the \emph{Heckelphon} (which will be used later on
						by Strauss in his \emph{Salom\'e}) and a considerable improvement
						of the \emph{contrabassoon} (which will be used in the
						\emph{Parsifal}).
        \end{column}
        \begin{column}{0.05\textwidth}
            \pgfimage[height=0.8\textheight]{\imagedir/heckelphon_big}
        \end{column}
        \begin{column}{0.1\textwidth}
            \pgfimage[height=0.8\textheight]{\imagedir/kontrafagott_big}
        \end{column}
    \end{columns}
\end{frame}

