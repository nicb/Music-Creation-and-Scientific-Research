\setcounter{ms}{0}
\refstepcounter{ms}
\begin{frame}
    \frametitle<+- | alert@+->{Expressive spatializazion (\arabic{ms})}

    \begin{columns}[T]
        \begin{column}{0.7\textwidth}
						\alert{In 2002, italian composer Adriano Guarnieri asked a more
						expressive spatial rendering of a trombone quartet within the
						performance of his opera \emph{Medea}.
					  The problem was studied within the Centro di Sonologia
						Computazionale of the University of Padova and an ``expressive
						spatialization system'', unique in its genre, was created.}
        \end{column}
        \begin{column}{0.3\textwidth}
            \pgfimage[width=\columnwidth]{\imagedir/guarnieri-02} 
        \end{column}
    \end{columns}
    
\end{frame}

\refstepcounter{ms}
\begin{frame}
    \frametitle{Expressive spatialization (\arabic{ms})}

    \begin{center}
        \begin{figure}
            \begin{center}
                \pgfimage[height=0.6\textheight]{\imagedir/eyesweb-processing}
            \end{center}
						\caption{\alert{\emph{Medea}'s ``expressive spatialization'' system (Amalia de G\"otzen)}}
            \label{fig:medea processing}
        \end{figure}
    \end{center}
    
\end{frame}
